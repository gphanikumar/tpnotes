\appendix

\chapter{Derivations}

% ---------------------------------------------------------------------------

\section{The quotient rule}
\label{quotientrule}

Adapted from section 2.62 of \cite{aris}.

Let $b_{i}$ be {\bf any} vector and $a_{ij}$ be a matrix of nine numbers ($i,j = 1,2,3$). If a relation could be found such that $ a_{ij} b_i = c_j $ is vector, then $a_{ij}$ is a tensor of order 2.

Consider a co-ordinate transformation with the tranformation matrix $T_{ij}$. Since $b$ and $c$ are vectors,
\begin{equation}
\begin{array}{ccc}
b_p^* = T_{pi} b_i & \mathrm{or} & b_i = T_{pi} b_p^* 
\end{array}
\end{equation}


$$ c_q^* = T_{qj} c_j $$

In the new co-ordinate system:
$$ a_{pq}^* b_p^* = c_q^* = T_{qj} c_j = T_{qj} a_{ij} b_i = T_{qj} a_{ij} T_{pi} b_p^* $$

$$ ( a_{pq}^* - T_{pi} T_{qj} a_{ij} ) b_p^* = 0 $$

Since $b_p^*$ is only the arbitrary vector $b_i$ in new co-ordinate system which is independent of $a_{ij}$ the only way the above equation can hold is if  

$$ a_{pq}^* = T_{pi} T_{qj} a_{ij}  $$

which is the definition for the entity $a_{ij}$ to be a tensor of order two.

%-------------------------------------------------------------------------



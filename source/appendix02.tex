\section{Symmetric tensors are diagonalisable}
\label{diagonalisable}

Adapted from section 2.5 of \cite{aris}.

If $a_{ij}$ is a tensor and $b_i$ is an arbitrary vector then by quotient rule, $c_j = a_{ij}b_i$ is a vector. For some $b_i$, the vector $c_i$ could be in the same orientation such that if $\lambda$ is the ratio of their magnitudes, $a_{ij} b_i = \lambda b_j = \lambda \delta_{ij} b_i$ . 

$$ \left( a_{ij} - \lambda \delta_{ij} \right) b_i = 0 $$

This is a set of three unknowns ($a_1$, $a_2$ and $a_3$) and for a solution to exist, $\lambda$ must satisfy the following equation:

$$ det \left( a_{ij} - \lambda \delta_{ij} \right) = 0 $$

$$ \lambda^3 - X \lambda^2 + Y \lambda - Z = 0$$

$$ X = a_{ii} = \mathrm{Trace}(a_{ij}) $$

$$ Y = a_{22}a_{33} - a_{23}a_{32} + a_{11}a_{33} - a_{13}a_{31} + a_{22}a_{11} - a_{12}a_{21} $$

$$ Z = det (a_{ij}) $$

$X$, $Y$ and $Z$ are three invariants of $a_{ij}$ under rotation of co-ordinate system.

The above is called the characteristic equation of the tensor $a_{ij}$ and the three values of $\lambda$ its characteristic values / latent roots / eigen values.

For the three eigen values $\lambda_{(p)}, p=1:3$, the corresponding characteristic vectors (${b_i}_{(p)}, p=1:3$ be denoted as $T_{j1}$, $T_{j2}$ and $T_{j3}$. 


\begin{equation}
T = \left(
\begin{array}{ccc}
{b_1}_{(1)} & {b_1}_{(2)} & {b_1}_{(3)} \\
{b_2}_{(1)} & {b_2}_{(2)} & {b_2}_{(3)} \\
{b_3}_{(1)} & {b_3}_{(2)} & {b_3}_{(3)} 
\end{array}
\right) = \left(
\begin{array}{ccc}
T_{11} & T_{12} & T_{13} \\
T_{21} & T_{22} & T_{23} \\
T_{31} & T_{32} & T_{33} \\
\end{array}
\right)
\end{equation}

\begin{equation}
\begin{array}{l}
a_{ij}T_{j1} = \lambda_1 T_{i1} \\
a_{ij}T_{j2} = \lambda_2 T_{i2}
\end{array}
\end{equation}

Multiply the first equation with $T_{i2}$ and the second equation with $T_{i1}$.

\begin{equation}
\begin{array}{l}
a_{ij}T_{j1} T_{i2} = \lambda_1 T_{i1} T_{i2} \\
a_{ij}T_{j2} T_{i1} = \lambda_2 T_{i2} T_{i1}
\end{array}
\end{equation}

Transpose the second equation (swap the indices $i$ and $j$ in second equation):

\begin{equation}
\begin{array}{l}
a_{ij}T_{j1} T_{i2} = \lambda_1 T_{i1} T_{i2} \\
a_{ji}T_{i2} T_{j1} = \lambda_2 T_{j2} T_{j1}
\end{array}
\end{equation}

If $a_{ij}$ is {\bf symmetric}, then

\begin{equation}
\begin{array}{l}
a_{ij}T_{j1} T_{i2} = \lambda_1 T_{i1} T_{i2} \\
a_{ij}T_{i2} T_{j1} = \lambda_2 T_{j2} T_{j1}
\end{array}
\end{equation}

ie.,

$$ a_{ij}T_{j1} T_{i2} = \lambda_1 T_{i1} T_{i2}  = \lambda_2 T_{j2} T_{j1} $$

Since $\lambda_1$ and $\lambda_2$ are distinct, the above equation can be true only if 

\begin{equation}
\begin{array}{l}
T_{i1} T_{i2}  = \delta_{ij}
\end{array}
\end{equation}

If we choose a co-ordinate transformation with the transformation matrix to contain the elements of $T_{ij}$ made out of the three eigen vectors,

$$ a_{pq}^* = T_{ip}T_{jq} a_{ij} = \lambda_p T_{ip}T_{iq} = \lambda_p \delta_{pq} $$

ie., $a_{pq}$ is diagonal. 


% ---------------------------------------------------------------------------

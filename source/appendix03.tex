\section{Levi-Civita tensor is isotropic}
\label{levicivitaisotropic}

Adapted from section 2.7 of \cite{aris}.

Since $\epsilon_{ijk}$ is a tensor of order 3, for an arbitrary co-ordinate transformation $T_{ij}$ ,


\begin{equation}
T_{ip} T_{jq} T_{kr} \epsilon_{ijk} = \left|
\begin{array}{ccc}
T_{1p} & T_{1q} & T_{1r} \\
T_{2p} & T_{2q} & T_{2r} \\
T_{3p} & T_{3q} & T_{3r}
\end{array}
\right|
\end{equation}

In the above determinant, if $p=q$ or $p=r$ or $q=r$, two columns being same will make the R.H.S. go to 0.
If $p$, $q$ and $r$ are cyclic (e.g., $1,2,3$) then the R.H.S. is the determinant of the transformation matrix itself which is 1.
If $p$, $q$ and $r$ are anti-cyclic (e.g., $1,3,2$), then the R.H.S. is the determinant of the transformation matrix (with two rows interchanged) which is -1.

These are the values defined for $\epsilon_{pqr}$. Hence the R.H.S. can be equated to $\epsilon_{pqr}^*$. Thus,

$$\epsilon_{pqr}^* = T_{ip} T_{jq} T_{kr} \epsilon_{ijk} $$

Since the above result is applicable for {\bf any} $T_{ij}$, $\epsilon_{ijk}$ is isotropic.

% -----------------------------------------------------------------------------------------


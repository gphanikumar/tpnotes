\section{Simplification of tensor properties for crystals}
\label{cubicsimplify}

Lars Onsager has proved (\cite{onsager1},\cite{onsager2}) using {\em microscopic reversibility} that for crystals with symmetries of order 3,4,6 etc., the thermal conductivity tensor is symmetric.

\begin{equation}
k_{ij} = \left(
\begin{array}{lll}
k_{11} & k_{12} & k_{13} \\
k_{12} & k_{22} & k_{23} \\
k_{13} & k_{23} & k_{33}
\end{array}
\right)
\end{equation}

For cubic crystals, the form of the tensors simplifies even further. Take co-ordinate system rotations about $\hat{x}_3$ by $90^0$ for which the transformation matrix is:

\begin{equation}
T = \left(
\begin{array}{lll}
0 & 1 & 0 \\
-1 & 0 & 0 \\
0 & 0 & 1
\end{array}
\right) 
\end{equation}


By definition, if $T$ is the transformation matrix of a co-ordinate axes rotation,

$$ k_{ij}^* = T_{pi} T_{qj} k_{pq} $$

Since the tensor in discussion is a property that should not change upon co-ordinate rotations that leave the crystal identical,
$$ k^* = k$$

Take $k_{11}$ and expand the above definition of tensor:

$$k_{11} = T_{i1}T_{j2}k_{ij} = T_{21}T_{21}k_{22} = k_{22}$$
or
$$k_{11} = k_{22}$$

Similarly,

$$k_{21} = T_{i2}T_{j1}k_{ij} = T_{12}T_{21}k_{12} = -k_{12} = -k_{21}$$
or
$$k_{21} = k_{12} = 0$$

Similarly rotating the co-ordinate system about the other two axes will show that all off diagonal terms of $k_{ij}$ vanish and all diagonal terms are same.

\begin{equation}
k_{ij} = \left(
\begin{array}{lll}
k_{11} & 0 & 0 \\
0 & k_{11} & 0 \\
0 & 0 & k_{11}
\end{array}
\right) = k \delta_{ij}
\end{equation}

Thus, only one value of $k$ is needed to completely specify the second order symmetric tensor property such as thermal conductivity of a cubic crystal.



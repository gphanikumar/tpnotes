\section{Cauchy's stress principle}
\label{cauchy}

Adapted from section 5.11 of \cite{aris}.

Let $\hat{n}$ be the unit outward normal at a point of the surface $S$ and $\sigma(\hat{n})$ the force per unit area exerted there by the material outside $S$. Then Cauchy's principle asserts that $\sigma(\hat{n})$ is a function of the position $x$, the time $t$ and the orientation $n$ of the surface element. Thus the total internal force exerted on the volume $V$ through its bounding surface $S$ is 

$$ \int_{S}{\sigma(\hat{n}) dS } $$

If $f$ is the external force per unit mass (e.g., $f=-g\hat{x}_3$), the total external force will be 

$$ \int_{V}{\rho f dV} $$

The principle of conservation of linear momentum asserts that the sum of these two forces equals the rate of change of linear momentum of the volume.

$$ \frac{d}{dt} \int_{V}{\rho v dV} = \int_{V}{\rho f dV} +  \int_{S}{\sigma(\hat{n}) dS } $$ 

If $V$ is a volume of a given shape with a characteristic dimension $d$ then $V \sim d^3$ and $S \sim d^2$. As we let $V$ shrink on a point but preserve the shape, the first two integrals decrease as $d^3$ where as the last will as $d^2$. So,

$$ \lim_{d \to 0}{\frac{1}{d^2}\int_{S}{\sigma(\hat{n}) dS}} = 0 $$

ie., the stresses are locally in equilibrium.



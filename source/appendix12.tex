\section{Stress tensor is symmetric}
\label{symstress}

Adapted from section 2.7 pg. 57 of \cite{sabersky}.

Consider stress at the center of a control volume of size $ \delta x_1 \times \delta x_2 \times \delta x_3$ as shown in the figure. The torque produced by the forces about an axis along $\hat{x}_3$ and through the center of gravity of the CV is 

$$ T = \sigma_{12} \delta x_1 \delta x_3 \delta x_2 - \sigma_{21}  \delta x_2 \delta x_3 \delta x_1 $$

or 

$$ T = ( \sigma_{12} - \sigma_{21} ) \delta x_1 \delta x_2 \delta x_3 $$

The torque may be equated to the product of angular acceleration ($\ddot{\alpha_3}$) and the moment of inertia taken about the previously mentioned axis ($\hat{x}_3$):


$$ ( \sigma_{12} - \sigma_{21} ) \delta x_1 \delta x_2 \delta x_3 = \frac{\rho}{12}  \delta x_1 \delta x_2 \delta x_3 (\delta x_1^2 + \delta x_2^2) \ddot{\alpha_3}  $$

or

$$ ( \sigma_{12} - \sigma_{21} ) = \frac{\rho}{12} (\delta x_1^2 + \delta x_2^2) \ddot{\alpha_3}  $$

As we shrink the CV to infinitesimal size, since the term in the bracket on the right goes to zero, if the L.H.S. were to remain finite $\ddot{\alpha_3}$ must blow up. It can be prevented only if L.H.S. is zero. ie., 

$$ \sigma_{12} = \sigma_{21} $$

or

The tensor $\sigma_{ij}$ is {\bf symmetric}.



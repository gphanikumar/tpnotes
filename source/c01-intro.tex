\chapter{Introduction to the course}

\section{Description}

{\bf MM2041} Transport Phenomena in Materials \\
Credits: 3 0 0 3

\begin{itemize}
\item Fundamentals of heat conduction, convection, radiation and their combined effect; steady and unsteady heat transfer in metallurgical processes, e.g. continuous casting, spray forming, solidifcation, extrusion etc.

\item Diffusion and its application in solid state materials processing, convective mass transfer in extraction processes, unsteady diffusion in finite and infinite bodies, diffusion and chemical reaction in porous and nonporous solid. 

\item Newton's law of viscosity, laminar flow problems related to metallurgy, general equation of continuity and motion, application of Bernoulli's equation in flow measuring devices and flow from ladles. 
\end{itemize} 

{\bf References:}

\begin{enumerate}
\item Transport Phenomenon in Metallurgy \\
G.H. Geiger and D.R. Poirier\\
Addison Wesley, 1973
\item Transport Phenomenon \\
R.B. Bird, W.E. Stewart and E.N. Lightfoot\\
John Wiley and Sons, 1994 
\item Transport Phenomena and Materials Processing\\
Sindo Kou \\
John Wiley and Sons. Inc., 1996 
\item Transport Phenomena in Materials Processing\\
D.R. Poirer and G.H. Geiger\\
TMS, Warrendale, PA, 1998
\end{enumerate} 

\pagebreak

{\bf MT538} Transport Phenomena in Metallurgical Processes\\
Credits: 3 0 0 3

\begin{itemize}
\item Types of reactors like plug flow, fluidised bed and their applications. Transport of heat and mass in sinter beds, fluidised beds, plug flow reactions as applied to metallurgical processes.

\item Convective and diffusion controlled mass transport. 

\item Modelling of metallurgical reactions, slag-metal reactions, heat treatment processes, sponge iron making, hydro metallurgical and pyro metallurgical processes in non-ferrous extraction

\end{itemize} 

{\bf References:}

\begin{enumerate}
\item Transport Phenomenon in Metallurgy \\
G.H. Geiger and D.R. Poirier\\
Addison Wesley, 1973
\item An Introduction to Transport Phenomena in Materials Engineering\\
David R. Gaskell\\
Macmillan, New York (1992)
\item Transport Phenomena and Materials Processing\\
Sindo Kou \\
John Wiley and Sons. Inc., 1996 
\item Transport Phenomena in Materials Processing\\
D.R. Poirer and G.H. Geiger\\
TMS, Warrendale, PA, 1998
\item Rate phenomena in process metallurgy\\
Julian Szekely and Nickolas J. Themells\\
John Wiley and sons, New York (1971) 
\end{enumerate} 

\section{Aim of the course}

\begin{list}{}{}
\item Introduce transport of heat, momentum and mass - in materials of our interest
\item Physics behind the transport phenomena
\item Range of applicability of the equations, initial and boundary conditions, solution methods
\item Similarity and relations between the three transport phenomena
\end{list}

\section{Learning objectives and Bloom's Taxonomy}

In this book, we use the revised Bloom's taxonomy as given by Anderson and co-workers~\cite{anderson}. The learning objectives will appear with the following tags. The action verbs and their definitions as in~\cite{anderson} are given below. An attempt is made to either use the action verb in the question or map each question in the book to the corresponding learning objective.

{\bf C1 : Remember} 
\begin{itemize}
\item Recognize / Identify : being able to correctly relate to knowledge in long term memory
\item Recall / Retrieve : being able to extract knowledge from long term memory
\end{itemize}

{\bf C2 : Understand} 
\begin{itemize}
\item Classify / Categorize / Subsume : Determining that something belongs to a category
\item Summarize / Abstract / Generalize : being able to express the take away
\item Interpret / Clarify / Paraphrase / Represent / Translate : changing from one form of representation to another
\item Exemplify / Illustrate / Instantiate : Finding a specific example or illustration of a principle or concept
\item Infer / Conclude / Extrapolte / Interpolate / Predict : being able to draw a logical conclusion
\item Explain / Construct : arrive at the cause and effect relationship
\item Compare / Contrast / Map / Match : being able to relate two (sets) of concepts, objects, ideas etc.
\end{itemize}

{\bf C3A : Apply} 
\begin{itemize}
\item Execute / carry out : apply a sequence of steps to a related and familiar task
\item Implement / Use : apply a sequence of steps to an unfamiliar task
\end{itemize} 

{\bf C3B : Analyze} 
\begin{itemize}
\item Differentiate / Discriminate / Distinguish / Focus / Select : being able to pick the desirable or relevant portions from the rest
\item Organize / Find / Integrate / Outline / Parse / Structure : being able to determine how different pieces fit with each other in a coherent structure of knowledge
\item Attribute / Deconstruct : being able to identify the underlying intent / point of view / value / bias
\end{itemize}

{\bf C3C : Evaluate} 
\begin{itemize}
\item Check / Coordinate / Detect / Monitor / Test : detecting inconsistencies in a process, determining effectiveness of a procedure or its implementation
\item Critique / Judge : detect inconsistencies against a criterion of evaluation, determine suitability of a procedure
\end{itemize}

{\bf C3D : Create} 
\begin{itemize}
\item Generate / Hypothesize : being able to come up with an alternative hypothesis based on a criterion
\item Plan / Design : Devise a procedure to accomplish a task
\item Produce / Construct : Invent a product
\end{itemize}

\section{Prelude}

In the course, we will use multiple notations to write down the equations. Consider the following alternate forms of the equation that states divergence of a vector $u$ is zero.

$$\vec{u} = u_1 \hat{x}_1 + u_2 \hat{x}_2 + u_3 \hat{x}_3 $$

$$\vec{\nabla} = \hat{x}_1 \frac{\partial}{\partial x_1} + \hat{x}_2 \frac{\partial}{\partial x_2} + \hat{x}_3 \frac{\partial}{\partial x_3} $$

\begin{eqnarray}
\frac{\partial u_1}{\partial x_1} + \frac{\partial u_2}{\partial x_2} + \frac{\partial u_3}{\partial x_3} = 0 \\
\vec{\nabla} \cdot \vec{u} = 0 \\
\nabla_i u_i = 0 \\
u_{i,i} = 0
\end{eqnarray} 

{\bf Continuum hypothesis:} The material is continuous in structure such that the physical quantities associated with the material within a given small volume are regarded as spread over uniformly over that volume.


The hypothesis is valid for most of the situations. In situations such as when the length scales are nanometric and the time scales are picometric, the hypothesis breaks down.

\section{Online videos}

Videos of lectures based on this course are freely available on youtube. The links for the videos and the mapping of the content are given below.

The youtube links can be accessed by prepending the string with {\tt https://youtu.be/}.

\begin{longtable}{|p{1cm}|p{8cm}|p{3cm}|p{2cm}|}
\hline
Lec & Topics & Link & Mapping \\
\hline
	01 & {\bf Subscript Notation - Part 1}: Subscript notation, Einstein summation convention,  use of comma for differentiation, inner and outer products, Kronecker delta, Trace of Matrix, use of Kronecker delta to replace free indices. & \href{https://youtu.be/7xNUBQax8Hk}{\tt 7xNUBQax8Hk} & Chapter~\ref{ch:subnot1} \\
\hline
	02 & {\bf Subscript Notation - Part 2} : Levi-Civita symbol / Permutation matrix, curl, cross product, determinant, double cross products and use of Kronecker delta to simplify products of Levi-Civita symbols, proving vector identities using subscript notation & \href{https://youtu.be/43\_Z57\_lzRY}{\tt 43\_Z57\_lzRY} & Chapter~\ref{ch:subnot2} \\
\hline
	03 & {\bf Coordinate Rotation} : Orthogonal coordinate system, handedness, transformation matrix for coordinate rotation and its properties, definition of tensors of different order. & \href{https://youtu.be/CaZTftBJD6I}{\tt CaZTftBJD6I} & Chapter~\ref{ch:coordtrans} \\
\hline
	04 & {\bf Introduction to Tensors}: Tensor quantities of different order, forms of isotropic tensors of different order, algebraic operations on tensors, symmetric and anti-symmetric tensors, inner and outer product of tensors, contraction theorem, quotient theorem. & {\tt FbnS5qsRkSg} & Chapter~\ref{ch:tensintro} \\
\hline
	05 & {\bf Symmetry of Properties}: Constitutive relationships with examples illustrating tensors of different order, Curie principle, Neumann principle, imposing symmetry of crystal on a tensor property and evaluating components, Diagonalization of symmetric tensors. & \href{https://youtu.be/UjhI0PwkSko}{\tt UjhI0PwkSko} & Chapter~\ref{ch:symprop} \\
\hline
	06 & {\bf Material Derivative}: Eulerian specification, material derivative, control volume approach, convention for flux, divergence theorem, velocity gradients and strain rate tensor, rate of dilation, definition of incompressible fluid, balance of mass and continuity equation in rectangular, cylindrical and spherical coordinate systems, application of continuity equation to determine missing components of incompressible fluid flow in rectangular coordinate system. & \href{https://youtu.be/weOlLUmy0oY}{\tt weOlLUmy0oY} & Chapter~\ref{ch:continuity} \\
\hline
	07 & {\bf Planar Flows}: Stream function in rectangular, cylindrical and spherical coordinate systems, vorticity, definition of irrotational flow, velocity potential, potential flows, elementary planar flows and their combinations, application to metallurgical problems. & \href{https://youtu.be/Q1daoeW7-3Q}{\tt Q1daoeW7-3Q} & Chapter~\ref{ch:planflow} \\
\hline
	08 & {\bf Reynolds Transport Theorem} : Concept map of derivation of Reynolds transport theorem, Deformation of fluid element due to velocity gradients, convected coordinates, Jacobian of transformation, rate of dilation in terms of Jacobian of transformation, material derivative of integral over control volume, Reynolds transport theorem, application to continuity equation and momentum. & \href{https://youtu.be/V-OoTDBNZK4}{\tt V-OoTDBNZK4} & Appendixes  \ref{variablechange}, \ref{materialderivative}, \ref{rtt}, \ref{ktt} \\
\hline
	09 & {\bf Derivation of Navier-Stokes equation - Part 1} : Concept map of derivation of Navier-Stokes equation, Newtons second law of balance of linear momentum to control volume approach, body forces, stress tensor and its symmetry, deviatoric stress, definition of pressure in terms of trace of stress tensor, velocity gradients and its components representing dilational strain rate, shear strain rate and rotational strain rate, linear constitutive relation between velocity gradient and deviatoric stress, definition of Newtonian fluid, Stokes assumption, Equation of motion, Navier-Stokes equation. & \href{https://youtu.be/Y3lcc5rCJ8E}{\tt Y3lcc5rCJ8E} & Chapter~\ref{ch:nsderive1} \\
\hline
	10 & {\bf Navier Stokes equations - Part 2} : Types of fluids based on relation between strain rate and shear stress, expressions for shear stress in rectangular and cylindrical coordinate systems, designation of different terms in Navier-Stokes equation, scaling of Navier-Stokes equation, Reynolds number, Navier-Stokes equations in 2D using stream function, Jacobian determinant, Special cases of Navier-Stokes equations under limiting conditions. & \href{https://youtu.be/elG1VXdLKO0}{\tt elG1VXdLKO0} & Chapter~\ref{ch:nsderive2} \\
\hline
	11 & {\bf Flow problem statements}: Assumptions used in flow problem definitions, Newtonian fluid, Incompressible fluid, constant properties, laminar regime, problem domain and boundaries, steady state, unidirectional flow, fully developed flow, boundary conditions, no slip condition, validity of analytical solutions. & \href{https://youtu.be/Z0ahHJ0ZUhE}{\tt Z0ahHJ0ZUhE} & Chapter~\ref{ch:flowprob} \\
\hline
	12 & {\bf Simple cases in fluid flow : rectangular coordinate system} : Convention for sign of shear stress, distribution of velocity and shear stress across domain for simple problems: film flow on an inclined plane, planar Couette flow, average flow rate, volume flow rate and mass flow rate, residence time and exposure time. & \href{https://youtu.be/nZgfzlCAm1E}{\tt nZgfzlCAm1E} & Chapter~\ref{ch:flowprob} \\
\hline
	13 & {\bf Simple cases in fluid flow : cylindrical coordinate system} : Pressure variation due to velocity components, unidirectional radial flow in cylindrical coordinate system, Poiseuille flow, average flow rate through a pipe, Hagen-Poiseuille equation, Axial flow through annular region, Couette flow, limiting case enabling planar flow assumption. & \href{https://youtu.be/pdIlZvocUbI}{\tt pdIlZvocUbI} & Chapter~\ref{ch:momcasescyl} \\
\hline
	14 & {\bf Pipe flow and porous medium}: Electrical analogue to pipe flow, definition of hydraulic radius, concept of superficial velocity through a porous medium, modeling porous medium as a bundle of tubes, Blake-Kozeny equation, Darcy's law, definition of Reynold's number for porous media, bed of spheres as a porous medium.& \href{https://youtu.be/qEYDWzyPyqo}{\tt qEYDWzyPyqo} & Chapter~\ref{ch:creepflow} \\
\hline
	15 & {\bf Simple cases in fluid flow : spherical coordinate system}: Concept map of derivation of Stokes equation, creeping flow over a sphere, Stokes equation, stream function solution, friction drag and form drag, Stokes law for terminal velocity. & \href{https://youtu.be/yFiZ2aGQi1k}{\tt yFiZ2aGQi1k} & Chapter~\ref{ch:creepflow} \\
\hline
	16 & {\bf Friction factors and correlations} : Definition of friction factor for internal and external flow, friction factor expressions from exact analytical solutions to Navier-Stokes equations, empirical correlations for simple cases in turbulent regime. & \href{https://youtu.be/tFk7ulqJxTE}{\tt tFk7ulqJxTE} & Chapter~\ref{ch:turbulent} \\
\hline
	17 & {\bf Energy Transport}: Enthalpy change without phase change, spontaneity of thermal equilibration using total entropy change, Fourier law for heat conduction, derivation of governing equation for heat conduction, boundary conditions for heat transfer, heat transfer coefficient. & \href{https://youtu.be/RY-JNULWk3g}{\tt RY-JNULWk3g} & Chapter~\ref{ch:energytransport} \\
\hline
	18 & {\bf Conduction  cases - steady state} : Steady state heat conduction across rectangular slab, cylindrical wall and spherical shell, point effect of diffusion, thermal resistances, electrical analogue to heat conduction, limiting case to approximate cylindrical and spherical cases to rectangular equivalents, steady state heat conduction across a composite wall. & \href{https://youtu.be/yS0tVTsFGI4}{\tt yS0tVTsFGI4} & Chapter~\ref{ch:heatcases1} \\
\hline
	19 & {\bf Conduction  cases - transient state} : Transient heat conduction in 1D, interface limited and diffusion limited heat transfer, Biot number, Fourier number, Lumped heat capacitance method, error function solution, application of error function to determine interface temperature in metallurgical problems such as casting, Chvorinov's rule. & \href{https://youtu.be/XB4wVJT0yd4}{\tt XB4wVJT0yd4} & Chapter~\ref{ch:heatcases2} \\
\hline
	20 & {\bf Convective heat transfer} : Effect of advection on heat transfer / convective heat transfer, Peclet number, limiting cases for use of upstream temperature / upwind scheme, heat transfer in a pipe with Poiseuille flow, Bulk temperature, Nusselt number, empirical correlations of Nusselt number for simple cases. & \href{https://youtu.be/49zn\_rY5HEQ}{\tt 49zn\_rY5HEQ} & Chapter~\ref{ch:convheat} \\
\hline
	21 & Mass Transfer overview & \href{https://www.youtube.com/watch?v=azXGH2kcDJU}{\tt azXGH2kcDJU} & \\
	\hline
	22 & Chemical equilibrium & \href{https://www.youtube.com/watch?v=c5T6bttIWSQ}{\tt c5T6bttIWSQ} & \\
	\hline
	23 & Reaction equilibrium & \href{https://www.youtube.com/watch?v=1LvAAmpr7fA}{\tt 1LvAAmpr7fA} & \\
	\hline
	24 & Species Balance equation & \href{https://www.youtube.com/watch?v=TuiURnxQ6h0}{\tt TuiURnxQ6h0} & \\
	\hline
	25 & Solute transfer modelling - Part 1 & \href{https://www.youtube.com/watch?v=BxGh2taiYXQ}{\tt BxGh2taiYXQ} & \\
	\hline
	26 & Solute transfer modelling - Part 2 & \href{https://www.youtube.com/watch?v=EBmByHHc6rA}{\tt EBmByHHc6rA} & \\
	\hline
	27 & Solute segregation profile - Part 1 & \href{https://www.youtube.com/watch?v=tBMw6-C4Hx4}{\tt tBMw6-C4Hx4} & \\
	\hline
	28 & Solute segregation profile - Part 2 & \href{https://www.youtube.com/watch?v=1429Vw7yk1c}{\tt 1429Vw7yk1c} & \\
	\hline
	29 & Problem statements & \href{https://www.youtube.com/watch?v=JWemMy2w1Gw}{\tt JWemMy2w1Gw} & \\
	\hline
	30 & Diffusion in solid state & \href{https://www.youtube.com/watch?v=NTXrsUwMgtA}{\tt NTXrsUwMgtA} & \\
	\hline
	31 & Transient diffusion in solid state & \href{https://www.youtube.com/watch?v=mYuLNGQR0FI}{\tt mYuLNGQR0FI} & \\
	\hline
	32 & Mass transfer in fluids & \href{https://www.youtube.com/watch?v=KfcZQmCT1k4}{\tt KfcZQmCT1k4} & \\
	\hline
	33 & Similarity Analysis & \href{https://www.youtube.com/watch?v=940uYw5DIBg}{\tt 940uYw5DIBg} & \\
	\hline
\caption{List of online videos and content description}
\label{tbl:links}
\end{longtable}

% ----------------- end of intro.tex ---------------------------

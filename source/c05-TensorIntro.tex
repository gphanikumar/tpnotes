\chapter{Introduction to Cartesian Tensors}
\label{ch:tensintro}

\section{Cartesian Tensors}

% learning objective
\begin{lo1}[Tensor introduction]
Recall the definitions of tensors and their order.
\end{lo1}


Many entities, such as those that have a physical meaning, are independent of the co-ordinate system we choose to represent them.  If $n$ is the dimension of the space we are concerned, the entities can be classifed - according to the number of components $(n^i)$ they would have - as {\bf tensors} of order $i$. 

\fbox{ %begin minipage frame
\begin{minipage}{4 in}
In this notes, when we say tensor, we imply Cartesian tensor. We don't want to get into covariant, contravariant and such things at this stage.
\end{minipage}
} % end of minipage frame



\subsection{Scalar}
Scalar is a tensor of order zero. Only one value would suffice to describe it at any location. Scalars are invariant under co-ordinate transformations.

{\bf Definition:} Scalar is an entity that remains invariant under all co-ordinate transformations.

Eg: Temperature $T$, Composition of - say - $A$ species $C_A$, Pressure in a fluid $p$, Density $\rho$, etc., 

\subsection{Vector}
Vector is a tensor of order one. It requires as many values / components as the dimension of the space (which is three for most of us) ie., a set of three components in a 3D space. 

{\bf Definition:} Vector is an entity that transforms the following way under a co-ordinate transformation: 
\begin{equation}
a^*_p = T_{pi}a_i
\end{equation} 

Eg: Velocity of fluid $\vec{u} = u_i$, Temperature gradient $\partial T \over \partial x_i$, Composition gradient $\partial C_A \over \partial x_i$, Displacement $u_i$, Electric Potential difference $P$, Electric Current $J_i$ etc.,

\subsection{Bisor}
Bisor is a tensor of order two. All tensors of order 2 or more are called as just {\bf tensors}. Specific names for tensors of order 2 and above are often considered old fashioned. A tensor of order 2 requires 9 components to describe it in general in 3D. 

{\bf Definition:} Tensor of order two is an entity that transforms the following way under a co-ordinate transformation: 
\begin{equation}
a^*_{pq} = T_{pi}T_{qj}a_{ij}
\end{equation} 

Eg: Stress $\sigma_{ij}$, Strain $e_{ij}$, Electrical Conductivity $\sigma_{ij}$, Electrical Resistivity $\rho_{ij}$, Diffusivity $D_{ij}$, Thermal Conductivity $k_{ij}$, Magnetic Permeability $\mu_{ij}$, Dielectric Permittivity $\epsilon_{ij}$, Thermal Expansion Coefficient $\alpha_{ij}$, Gyration Tensor $g_{ij}$

\subsection{Trisor}

Trisor is a tensor of order three. 27 components! 

{\bf Definition:} Tensor of order three is an entity that transforms the following way under a co-ordinate transformation: 
\begin{equation}
a^*_{pqr} = T_{pi}T_{qj}T_{rk}a_{ijk}
\end{equation} 

Eg: Piezoelectric coefficient $e_{ijk}$, 

\subsection{Tetror}

Tetror is a tensor of order four. 81 components! 
\begin{equation}
a^*_{pqrs} = T_{pi}T_{qj}T_{rk}T_{sl} a_{ijkl}
\end{equation}

Eg: Elastic Modulus $c_{ijkl}$

\subsection{Tensor}

\index{Tensor}

In general, a tensor is an entity that follows the following generalised rule for any co-ordinate transformation. The number of indices of $a$ indicate the order of the tensor which is the same as the number of times $T$ occurs in the expression.

\begin{equation}
\boxed{a^*_{pqrst...} = T_{pi}T_{qj}T_{rk}T_{sl}T_{tm}... a_{ijklm...} }
\end{equation}

% learning objective
\begin{lo2}[Tensor Introduction]
Validate a tensorial expression and identify the tensorial order of a parameter
\end{lo2}

Here we must bring in some examples.


\section{Operations on tensors}

\index{Tensor operations}

\begin{itemize}

\item Sum and difference of two tensors (of same order) is also a tensor. In the following example, if $b_{ij}$ and $c_{ij}$ are tensors, then $a_{ij}$ and $d_{ij}$ are also tensors.
$$ a_{ij} = b_{ij} + c_{ij} $$
$$ d_{ij} = b_{ij} - c_{ij} $$

\index{Dyadic product}
\item An outer product of a tensor of order $m$  with a tensor of order $n$ will give a tensor of order $m+n$. Outer product is also referred to as dyadic product.
$$a_{ijk} = b_{ij}c_{k}$$

\item An inner product of a tensor of order $m$ with a tensor of order $n$ will give a tensor of order $|m-n|$.
$$a_{i} = b_{ijk}c_{jk}$$

\index{Contraction theorem}
\item {\bf Contraction theorem}\label{contraction} \index{Contraction theorem} : If $a_{ijkl..}$ is a tensor of order $m$, then the entity obtained by repeating any two subscripts is a tensor of order $m-2$. Eg., if $d_{ijkl}$ is a tensor of order 4, then by repeating two of its indices (say, third and fourth), the entity obtained $a_{ij}$ is a tensor of order $4-2=2$.
$$ a_{ij} = d_{ijkk}$$


A corollary to the above theorem is obtained when we take a tensor of order two and repeat the indices. i.e., Trace of a second order tensor $a_{ii}$ is tensor of order zero or scalar or invariant across co-ordinate transformations.


\item {\bf Quotient law of tensors} \index{Quotient law}: If there is an entity representable by (subscript notation as) $a_{ij}$ relative to any cartesian co-ordinate system and if $a_{ij}b_i$ is a {\em vector} where $b_i$ is {\em any arbitrary vector} then $a_{ij}$ is a {\em tensor} of order two. Proof is given in section \ref{quotientrule}.

Combining with the theorem on outer product, quotient law can be extended to tensors of higher order.


\end{itemize}


\section{Types of tensors}

\subsection{Symmetric Tensor}
\index{Tensor, symmetric}

$a$ is a {\em symmetric} tensor if $a_{ij} = a_{ji}$

{\bf Theorem}: For every second order symmetric tensor, there exists a co-ordinate system relative to which, the matrix of the components of the tensor is diagonal \footnote{This theorem has important implications when applied to the stress tensor in the mechanical behaviour of materials. The diagonal terms are called principal stresses.}. Proof is given in section \ref{diagonalisable}.

\subsection{Anti-symmetric Tensor}
\index{Tensor, anti-symmetric}

$a$ is a {\em skew-symmetric} or {\em anti-symmetric} tensor if $a_{ij} = -a_{ji}$

The vector formed by the components of an anti-symmetric tensor:

$$
\omega_k = \left( 
\begin{array}{l}
\omega_1\\
\omega_2 \\ 
\omega_3
\end{array} 
\right)
$$

$$ \Omega_{ij} =  \left(
\begin{array}{ccc}
0 & \omega_3 & -\omega_2 \\
-\omega_3 & 0 & \omega_1 \\
\omega_2 & -\omega_1 & 0
\end{array} 
\right)
$$

$$\omega_k = \frac{1}{2} \epsilon_{kij} \Omega_{ij} $$
Also,
$$\Omega_{ij} = \epsilon_{ijk}\omega_{k}$$

\index{Dual tensors}
The two tensors $\omega_k$ and $\Omega_{ij}$ are called dual tensors.

{\bf Theorem}: Every second order tensor is expressible as a sum of a symmetric tensor and an anti-symmetric tensor. 

\begin{equation}
a_{ij} = \frac{1}{2}\left( a_{ij}+a_{ji}\right) + \frac{1}{2}\left( a_{ij}-a_{ji}\right) 
\end{equation} 



% ---------------------------------------------------------------------

% learning objective
\begin{lo2}[Tensor introduction]
Prove an expression to be a tensor of appropriate order using the definition of a tensor
\end{lo2}

% ---------------------------------------------------------------------

\begin{question}
Prove that gradient of a continuous scalar function is a vector. 
\begin{equation}
a_i = \nabla_i\phi = \frac{\partial \phi}{\partial x_i}
\end{equation} 
\end{question}
\begin{solution}[print]
\end{solution}

% ---------------------------------------------------------------------

\begin{question}
Prove that the magnitude of a vector is invariant under co-ordinate transformations.
\end{question}
\begin{solution}[print]
\end{solution}

% ---------------------------------------------------------------------

\begin{question}
Prove that $\delta_{ij}$ is (an isotropic) tensor of order two.
\end{question}
\begin{solution}[print]
\end{solution}

% ---------------------------------------------------------------------

\begin{question}
Prove that if $u_i$ and $v_j$ are vectors, then the dyad $a_{ij} = u_i v_j$ is a tensor of order two.
\end{question}
\begin{solution}[print]
\end{solution}

% ---------------------------------------------------------------------

\begin{question}
(Exercise 2.42.1 of \cite{aris}) Prove that for any vector $\vec{a}$, $\epsilon_{ijk}a_k$ are the components of a second order tensor.
\end{question}
\begin{solution}[print]
\end{solution}

% ---------------------------------------------------------------------

\begin{question}
(Exercise 2.42.3 and 2.44.2 of \cite{aris}) If $r^2 = x_k x_k$ and $f(r)$ is any twice differentiable function, show that the following nine derivatives are components of a tensor.
$$\left[ f''(r) -{f'(r) \over r} \right] {x_i x_j \over r^2} + {f'(r) \over r} \delta_{ij} $$.
Show also that the trace of that tensor is the following.
$${1 \over r^2}{d \over dr} \left( r^2 {df \over dr} \right)$$
\end{question}
\begin{solution}[print]
\end{solution}

% ---------------------------------------------------------------------

\begin{question}
Prove that strain rate or velocity gradient is a tensor of order two.
$$ e_{ij} = {\partial u_i \over \partial x_j}$$
\end{question}
\begin{solution}[print]
\end{solution}

% ---------------------------------------------------------------------

\begin{question}
If $\sigma_{ij}$ is a tensor, prove that $\sigma_{ii}$, the trace of the matrix of $\sigma_{ij}$ is invariant under co-ordinate transformations. \footnote{This quantity has a special meaning in mechanical behaviour of materials.}
\end{question}
\begin{solution}[print]
\end{solution}

% ---------------------------------------------------------------------

\begin{question}
Prove that $\epsilon_{ijk}$ is a tensor of order three. Clue: Use the subscript notation for determinant of the transformation matrix. When two rows of a matrix are same, the determinant vanishes.
\end{question}
\begin{solution}[print]
\end{solution}

% ---------------------------------------------------------------------

\begin{question}
Prove that the dyad $\delta_{ij}\delta_{kl}$ is (an isotropic) tensor of order four. Considering that $\delta_{ij}$ is an isotropic and hence a symmetric tensors, how many combinations of the four indices can you get so that you can arrive at possible isotropic tensors of order four.
\end{question}
\begin{solution}[print]
\end{solution}

% ---------------------------------------------------------------------

\begin{question}
If $a_{ij}$ and $b_{kl}$ are tensors of order two, prove that the dyad $a_{ij} b_{kl}$ is a tensor of order four. Prove that $a_{ij} b_{jk}$ is a tensor of order two. Also, that $a:b = a_{ij}b_{ji}$ is a scalar.
\end{question}
\begin{solution}[print]
\end{solution}

% ---------------------------------------------------------------------

\begin{question}
Use the Neumann's principle to prove that the number of components necessary to describe the thermal conductivity of a tetragonal crystal is two. You can assume that the thermal conductivity is a symmetric tensor to start with, thanks to Onsager's theory~\cite{onsager1, onsager2}.
\end{question}
\begin{solution}[print]
\end{solution}


\begin{center}
{\bf \Large Tensor Properties}
\end{center}

\begin{enumerate}{}{}

\item If $a_{ij}$ and $b_{ij}$ are two tensors of order 2, then write the order / rank and list the free indices for each of the following quantities.

\begin{tabular}{| c | c | m{1 in} | m{1 in} |}
\hline
 Sl. & Expression & Order / Rank & Free Indices \\[1 cm]
\hline
1 & $a_{ij} b_{jk}$ & & \\[1 cm]
\hline
2 & $a_{ij} b_{ik}$ & & \\[1 cm]
\hline
3 & $a_{ij} b_{kj}$ & & \\[1 cm]
\hline
4 & $a_{ij} b_{ij}$ & & \\[1 cm]
\hline
5 & $a_{ij} b_{ji}$ & & \\[1 cm]
\hline
6 & $a_{ji} b_{ij}$ & & \\[1 cm]
\hline
7 & $a_{ij} b_{ki}$ & & \\[1 cm]
\hline
8 & $a_{ij} b_{kl}$ & & \\[1 cm]
\hline
\end{tabular}

\item In the above list of expressions, which of the above qualifies as a matrix multiplication the way you knew it?

\item Consider the above expressions are contraction operations on $d_{ijkl}$ where $d_{ijkl} = a_{ij} b_{kl}$. List which expression is a contraction operation over which of the indices of $d$.

\item In the above list of expressions, are all the second order tensorial quantities identical?

\item The elements of a vector $\vec{a}$ in the new coordinate system after rotation from an old coordinate system are given as follows. Here, $T_{pi}$ indicates the transformation matrix corresponding to the rotation of axes. 
$$ a^*_p = T_{pi} \, a_i$$ 
Show that the components of this vector $\vec{a}$ in the old coordinate system in terms of the new coordinate system can be given by the following equation.
$$ a_i = T_{pi} \, a^*_p$$ 

\item If $b_i$ is any vector and $a_{ij}$ is a matrix of nine numbers and if a relation could be found such that $a_{ij} b_i = c_j$ is always a vector, then show that $a_{ij}$ is a tensor.\\
Hint: Start with definition of $b_i$ and $c_j$ being vectors. Apply the relation connecting them across a coordinate transformation. The result you obtain is called as the Quotient Rule.

\item Show that $\delta_{ij}$ is an isotropic tensor of order 2. That is, the elements of $\delta_{ij}$ do not change for {\bf any} arbitrary rotation of the coordinate system.\\
Hint: Use the subscript replacement ability of Kronecker delta and the orthogonality condition for a transformation matrix.

\item Show that $\epsilon_{ijk}$ is an isotropic tensor of order 3.\\
Hint: Use the formula for determinant of a matrix using the permutation symbol.

\item Show that $\delta_{ij} \delta_{kl}$ is an isotropic tensor of order 4. Show that the other two combination of indices will also give the same conclusion. Show that a linear combination of these three fourth order isotropic tensors is also an isotropic tensor of order four.

\end{enumerate}


{\bf To ponder / home work}

\begin{enumerate}
\item What can you recall about eigenvectors and eigenvalues for a $3 \times 3$ matrix?
\item If $A$ is a $3 \times 3$ matrix and $\delta_{ij}$ is the identity matrix, write down the determinant of the matrix $\left| A - \lambda \delta_{ij} \right|$. The cubic polynomial in $\lambda$ you obtain is called the characteristic polynomial and the equation that equals this polynomial to zero is called {\bf secular} equation or characteristic equation of a tensor of order two.
\index{Secular equation}
\index{Characteristic equation}
\item Collate the coefficients of $\lambda^3$, $\lambda^2$ and $\lambda$ in the above expression.
\item Write each of these coefficient expressions in subscript notation.
\item Is it right to say that each of these coefficients are invariant under rotation of coordinate system?
\end{enumerate}

% learning objective
\begin{lo2}[Tensor components]
\item Determine the components of a second order tensor after rotation of coordinate systems given the transformation matrix
\end{lo2}

Give some examples here

% learning objective
\begin{lo2}[Tensor components]
\item Determine unknown components of a tensor of second order by using the invariants
\end{lo2}

Give some examples here

\section{Summary}

\begin{enumerate}
\item 
\item 
\end{enumerate}

% --------- end of TensorIntro.tex --------------------------------

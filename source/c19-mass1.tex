\chapter{Mass Transfer}

\section{Introduction}

In this section, you will see that balance of mass taking individual species
into consideration will lead to the continuity equation. Also, the meaning of
velocity $\vec{u}$ used for the advective term in the earlier chapters in
relation to the velocities of individual species will also be evident.
Diffusional flux will be expressed using the difference in the velocity of a
species compared to that of the center of mass.

\fbox{ %begin minipage frame
\begin{minipage}{4 in}
In this chapter, the subscripts refer to the species and/or reactions. They should not be confused with subscript notation.
\end{minipage}
} % end of minipage frame

\section{Organization of topics}

Mass transfer topics fall into the following four major categories.
\begin{itemize}
\item \textbf{Solid state diffusion}: discussed mainly in the context of
physical metallurgy.
\item \textbf{Convective mass transfer}: when flux is more important to arrive at overall solute balances.
\item \textbf{Reaction with a generation term}: when solute is generated or
consumed at a location because of a reaction. The rate of formation of a certain
product is of interest here. We can use balance of fluxes taking into account
the fact that more than one species are involved in the transport.
\item \textbf{Solute redistribution}: as it occurs during partitioning during a phase transformation. For example, during solidification. Solute element partitioning is important not only for segregation of solute elements but also to understand formation of porosity due to dissolved gases.
\end{itemize}

Unlike the parameters used for Momentum transfer (Velocity) and heat transfer
(Temperature), the parameters used for mass transfer are varied in their
physical meaning as well as in their units.

\begin{itemize}
\item Concentration (mass per unit volume): default variable for us
\item Atom/Mole fraction or percentage or ppm: popular in physical metallurgy
\item Weight fraction or percentage or ppm: a practicable quantity used in
process metallurgy
\item Partial pressure: useful unit when dealing with gases
\item Fugacity
\item Activity
\end{itemize}

In our discussion, unless otherwise specified, we deal with the concentration as
our variable and convert the rest of the quantities to concentration when
necessary. 

In addition, we must also specify for which of the species are we writing down
the balance equation. Meaning, instead of $C$ as our variable, really we have
$C_i$ as our set of variables where $i$ goes over the range of number of species
participating in the mass transfer. Unless specified, we will be dealing with a
situation where the balance is written down only for one species.


\section{Chemical equilibrium}

We first introduce the idea of chemical equilibrium here with inputs from Metallurgical thermodynamics. The role of chemical potential is to be clarified with respect to things that happen in metallurgical scenario such as carburization as well as precipitation. Evening out of solute concentrations is to be shown as a special case of ideal solutions. 


The different variables that describe the content of solute are to be listed with their significance. The choice of solute concentration and conversions across are to be listed.


The subtle difference in chemical potential and diffusion potential and the difference between slope and intercept of $G_m$ versus $X_B$ plots to be explained. How these things are reconciled in binary and multi-component systems to be explained.


The need for definition of velocity in continuum and its reconciliation with species velocities is to be raised here.

Also we should introduce the concept of vapour pressure as function of temperature here.


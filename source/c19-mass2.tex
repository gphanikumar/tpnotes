\section{Species balance equation}

The derivation that follows is based on what is found in the first few pages of
the book by de Groot and Mazur~\cite{degroot}.

The symbols used are as follows: 
\begin{itemize}
 \item $\rho_k$ is the density of species $k$ in \si{\kilo\gram\per\meter\cubed}. This is the
default unit of concentration unless otherwise specified in most of the
subjects.
\item $\vec{u}_k$ is the velocity of the species $k$.
\item $N$ is the total number of species in the domain considered.
\item $R$ is the total number of reactions that take place in the domain.
\item In $j^\text{th}$ reaction, the rate of production of species $k$ is given
by $\kappa_{jk} Q_j$.
\item $Q_j$ is the chemical reaction rate of $j^\text{th}$ reaction in
\si{\kgpmcps}. 
\item The coefficients $\kappa_{kj}$ are taken as positive for the product
species and negative for the reacting species in the $j^\text{th}$ reaction.
\end{itemize}

The total density $\rho$ at any location is given by:

$$ \rho = \sum_{k=1}^{N}{\rho_k} $$

%%%%%%%%%%%%%%%%%%%%%%%%%%%%%%%%%%%%%%%%%%%%%%%%%%%%%%%%%%%%%%%%%%%%

\fbox{ %begin minipage frame
\begin{minipage}{6 in}
As an example, take the reduction of wustite by hydrogen to produce iron. Let
this be the $j^\text{th}$ reaction.
$$ \text{FeO} + \text{H}_2 \rightarrow \text{Fe} + \text{H}_2\text{O} $$

Using the molar weights in \si{\gram\per\mole} and a subscript $k$ to indicate
the number of the species - total $N$ being 4 here. 


$MW_1 \equiv MW_\text{FeO} =71.84$ \\
$MW_2 \equiv MW_{\text{H}_2} = 2$ \\
$MW_3 \equiv MW_\text{Fe} = 55.84$ \\
$MW_4 \equiv MW_{\text{H}_2\text{O}} = 18$ \\

$$\kappa_{j1} =  -{MW_1 \over MW_1 + MW_2} $$

We can write the equation with the coefficients as:

$$\SI{0.973}{\gram} \, \text{FeO} + \SI{0.027}{\gram} \, \text{H}_2 \rightarrow
\SI{0.756}{\gram} \, \text{Fe} + \SI{0.244}{\gram} \, \text{H}_2\text{O} $$ 

So that, $\kappa_{j1} = -0.973$, $\kappa_{j2} = -0.027$, $\kappa_{j3} = 0.756$
and $\kappa_{j4} = 0.244$.

Conservation of mass requires the following:
$$ \kappa_{j1} + \kappa_{j2} + \kappa_{j3} + \kappa_{j4} = 0 $$
\end{minipage}
} % end of minipage frame

%%%%%%%%%%%%%%%%%%%%%%%%%%%%%%%%%%%%%%%%%%%%%%%%%%%%%%%%%%%%%%%%%%%%

We notice that in the $j^\text{th}$ reaction the following equation implies
conservation of mass.

$$ \sum_{k=1}^{N}{\kappa_{jk}} = 0$$

Let $s_{kj}$ be the rate of production of $k^\text{th}$ species due to
$j^\text{th}$ reaction. 
The total production rate of $k^\text{th}$ species is given by:
$$ q_k = \sum_{k=1}^{N}{s_{kj}} $$

We can note that for each reaction $j$, $s_{jk} = \kappa_{jk}Q_j$. Thus,

$$ q_k = \sum_{k=1}^{N}{\kappa_{jk}Q_j} $$

Let $\vec{j}_k$ be the flux of species $k$ across the faces of the control
volume.

Consider a control volume with $dV$ as the volume and $dS$ as the surface area
in the domain fixed in space. $\hat{n}$ is the normal to each surface element.
The control
volume contains all the $N$ number of species with their respective
concentrations given by $\rho_k$. The change in the concentration of any
$k^\text{th}$ species is either due to an inward flux of that species from the
surface of the control volume or due to its generation from any of the $R$
reactions that are possible in the domain. Thus, we can write the balance of
mass for every $k^\text{th}$ species in the domain as follows:

$$ \int_{V}{ {\partial \rho_k \over \partial t} dV} = - \int_{S}{ \vec{j}_k
\cdot
\hat{n} dS} + \int_{V}{q_k dV} $$

We can express the flux of a species using the velocity of that species as

$$\vec{j}_k = \rho_k \vec{u}_k$$

Substituting,

$$ \int_{V}{ {\partial \rho_k \over \partial t} dV} = - \int_{S}{ \rho_k
\vec{u}_k \cdot \hat{n} dS} + \int_{V}{\sum_{j=1}^{R}{\kappa_{jk}Q_j} dV} $$

Using the Gauss theorem to convert the surface integral to volume integral, we
obtain

$$ \int_{V}{ {\partial \rho_k \over \partial t} dV} = - \int_{V}{ \vnabla \cdot
\left(\rho_k \vec{u}_k \right) dV} + \int_{V}{\sum_{j=1}^{R}{\kappa_{jk}Q_j} dV}
$$

Since this is valid for every control volume at every location in the domain, we
can write:

\begin{equation}
\label{rhokuk1}
\boxed{
 {\partial \rho_k \over \partial t}  = - { \vnabla \cdot \left(\rho_k \vec{u}_k
\right) } + \sum_{j=1}^{M}{\kappa_{jk}Q_j}
}
\end{equation}

\subsection{Continuity Equation}
 
We can now retrieve continuity equation that expresses the balance of mass from
the above species balance equation.


The above equation is also true for every species. So we can sum it up over all
the $N$ species.

$$ \sum_{k=1}^{N}{\partial \rho_k \over \partial t}  = - \sum_{k=1}^{N}{ \vnabla
\cdot \left(\rho_k \vec{u}_k \right) } +
\sum_{k=1}^{N}{\sum_{j=1}^{R}{\kappa_{jk}Q_j} } $$

Re-arranging the last term by taking $Q_j$ common and using the definition of
total density:

$$ {\partial \rho \over \partial t}  = - \sum_{k=1}^{N}{ \vnabla \cdot
\left(\rho_k \vec{u}_k \right) } + \sum_{j=1}^{R} {Q_j
\sum_{k=1}^{N}{\kappa_{jk} } } $$

Recognizing that the sum of $\kappa_{jk}$ over any $j^\text{th}$ reaction is
zero:

$$ {\partial \rho \over \partial t}  = - \sum_{k=1}^{N}{ \vnabla \cdot
\left(\rho_k \vec{u}_k \right) } $$

We now define the {\em barycentric velocity} or the {\em velocity of the centre
of mass} as follows:

$$ \vec{u} = { \sum_{k=1}^{N}{\rho_k \vec{u}_k} \over \sum_{k=1}^{N}{\rho_k}} =
{1 \over \rho} \sum_{k=1}^{N}{\rho_k \vec{u}_k} $$

Substituting this definition, the balance of mass comes to a form that is
familiar from the continuity equation:

$$ {\partial \rho \over \partial t}  = - \vnabla \cdot \left( \rho \vec{u}
\right) $$

Using the definition of substantial / material derivative,

$$ {\partial \rho \over \partial t} + \left( \vec{u} \cdot \vnabla \right) \rho 
= {D \over Dt} \rho =  - \rho \left( \vnabla \cdot \vec{u} \right) $$

\subsection{Diffusional flux}

We define {\it Diffusional flux} of a species as due to difference in the
velocity of the species from the velocity of the centre of mass.

We revisit equation~\ref{rhokuk1} by substituting {\em diffusional flux} of the
$k^\text{th}$ species as defined below:

\begin{equation}
 \label{diffluxdef}
\boxed{
\vec{J}_k = \rho_k \left( \vec{u}_k - \vec{u} \right) = \rho_k \vec{u}^*_k
}
\end{equation}


Here, $\vec{u}^*_k$ is the velocity of the species {\bf relative} to the centre of mass.


$$ {D \rho_k \over D t} = - \rho_k \vnabla \cdot \vec{u} - \vnabla \cdot
\vec{J}_k + \sum_{j=1}^{M}{ \kappa_{jk} Q_j } $$

Defining mass fraction $c_k = {\rho_k / \rho}$, we can write the following
balance of mass for each of the $k^\text{th}$ species:

\begin{equation}
 \label{mconv1}
\boxed{
\rho {D c_k \over D t} = - \vnabla \cdot \vec{J}_k + \sum_{j=1}^{M}{ \kappa_{jk}
Q_j } 
}
\end{equation}

Summing up equation~\ref{diffluxdef} over all species, we can conclude that:

$$ \sum_{k=1}^{N}{\vec{J}_k} = 0 $$

In other words, of the $N$ equations for the fluxes, only $N-1$ are independent.

\subsection{Ficks Law}

We now seek a linear constitutive relation to determine $\vec{J}$. Refer to the
section 43.14 of Feynman Lecture Series for more discussion at this point. Flux
of a species is usually expressed as a product of composition of that species
and the velocity of that species. Diffusional flux, however, is expressed as a product of composition and the velocity of the species relative to the center of mass.

$$ \vec{J}_k = (\vec{u}_k - \vec{u}) \rho_k = \vec{u}^*_k \rho_k $$


If velocity of a species relative to the centre of mass is the effect, 
then the cause is a gradient in chemical potential - meaning that atoms move to
reduce the free energy of the system. The quantity that connects  a cause and 
effect is a material property and one can use relevant theories to deduce 
the minimum number of entities necessary to represent that property.

$$ \vec{u}^* = - M \vec{\nabla} \mu_k $$

In solid state diffusion, typically one considers that the center of mass is not moving ie., $\vec{u} = 0$ or 
$\vec{u}_k = \vec{u}^*_k$. Thus the typical form in which the above equation is often seen is:

$$ \vec{u}_k  = - M \vec{\nabla} \mu_k $$

$M$ is called the \textbf{mobility tensor}. Using Onsager's reciprocal
relations~\cite{onsager1, onsager2}, one can say that $M$ is a symemtric tensor.
It is isotropic for isotropic media. For our discussion, we are usually limited
to liquids, gases, polycrystalline materials that can be approximated to
istotropic media.

$$ \vec{J}_k = -M \rho_k \vec{\nabla} \mu_k $$

Using an ideal solution approximation,

$$\mu = \mu_0 + RT \ln {X\gamma} = \mu_0 + RT \left( \ln X  + \ln \gamma
\right)$$

Where, $X$ is mole fraction and is proportional to $C$ and $\gamma$ is the
activity of the species in the solution. 
We now look at the form of diffusion equation in one dimension rectangular coordinate system for simplicity.

$$ -M \rho_k {\partial \mu \over \partial x} = - M \rho_k {\partial \mu \over \partial
\rho_k}{\partial \rho_k \over \partial x} = -M{\partial \mu \over \partial \ln \rho_k}{\partial
\rho_k \over \partial x} $$

Since composition can be related to atom fraction $X_k$ as 

$$ \rho_k = X_k M_w $$ 

$${\partial \mu \over \partial \ln \rho_k} = {\partial \mu \over \partial \ln X_k}$$

$$ M{\partial \mu \over \partial \ln X_k} = MRT \left( 1 + {\partial \ln \gamma
\over \partial \ln X_k} \right) = D_k $$

Where, $D_k$ is the diffusivity of the solute $k$ in the solution. Thus, the Fick's
first law of solute diffusion goes as
\begin{equation}
\boxed{
  \vec{J}_k = -D_k {\partial \rho_k \over \partial x}
}
\label{ficks-1}
\end{equation}

\subsection{Diffusion equation}

Substituting the same in the balance equation,

\begin{equation}
{\partial \rho_A \over \partial t} + \left( \vec{u} \cdot \vec{\nabla} \right) \rho_A =
\vec{\nabla} \cdot \left( D {\partial \rho_A \over \partial x} \right) + \sum_{j=1}^{r}{\kappa_{Aj}Q_j}
\end{equation}

Assuming the $D$ is constant over the range of variation of $\rho_A$, absence of any
generation term and that the domain is solid and the species A is dilute enough that $\rho_A$ is very small corresponding to $\rho$, we get the Fick's second law of
solute diffusion

\begin{equation}
\boxed{
{\partial \rho_A \over \partial t} = D \nabla^2 \rho_A
}
\end{equation}

This is called as the Diffusion equation or Fick's second law of diffusion.

\subsection{Diffusivity}

Unlike the relevant properties for momentum and heat transfer, the kind of
diffusivity to be used depends on the situation of mass transfer problem.
Following are some examples.

\begin{itemize}
\item \textbf{Diffusivity in gas}: Using kinetic theory of gases,
$$D_{AA} = {1 \over 3} \lambda u$$
Where, $\lambda$ is mean free path and $u$ is the average velocity of the atoms
in the gas.
$$\lambda = {k_B T \over \sqrt{2} \pi \sigma_A^2 \rho}$$
$$ u = \sqrt{8 k_B N T \over \pi M_A}$$
Thus, the diffusivity in gas goes as $T^{3/2}$.

\item \textbf{Diffusivity in gas entrapped in a pore}: In case the gas is
entrapped in a pore, instead of the mean free path from the kinetic theory of
gases, we are supposed to use the pore diameter. This means the temperature
dependence of Diffusivity goes as $T^{1/2}$.

\item \textbf{Liquid Diffusivity}: Using the Einstein equation,
$$k_B T = 6 \pi r_A \mu_B D_{AB}$$

\item \textbf{Solid Diffusivity}: Arrhenius variation

$$ D  = D_0 e^{-Q \over RT} $$

The activation energy for Diffusion in solid $Q$ depends on the way diffusion
takes place and following are some situations where they are different:

\begin{itemize}
\item Diffusion aided by non-equilibrium defect concentration eg., due to
quenched-in vacancies
\item Pipe diffusion aided by dislocations
\item Grain boundary aided diffusion
\item Surface diffusion
\item Stress induced diffusion
\item Bulk diffusion
\end{itemize}

\end{itemize}



